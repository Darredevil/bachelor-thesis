\chapter{Architectural Overview}
\label{chapter:architecture}

In the previous chapter the Radare2 Framework was presented, providing an overview of it. The following sections present the new modules introduced as well as how they interact with the framework. A diagram of the architecture can be seen in \labelindexref{Figure}{img:arch}.

\abbrev{CLI}{Command Line Interface}
\fig[scale=0.45]{src/img/Architecture-v3.png}{img:arch}{Architectural Overview}

As can be seen in \labelindexref{Figure}{img:arch} there are four new modules introduced:
\begin{itemize}
	\item Gadget Classification Module
		% \subitem This module provides the semantic analysis and classification for each gadget. The gadgets are generated using the search feature already implemented in the Core and fed to the module.
	\item SMTLib2 Parser
	\item SMT Solver
	\item ROP Payload Generator
\end{itemize}

% \todo{Explain a bit of pipeline between modules and make a diagram \\ Repaint with light grey the modules introduced by me in Fig 6.1}

In \labelindexref{Figure}{img:pipeline} we can see a pipeline representation of the modules. This illustrates in more detail how the data flows through the system from the first step of feeding the executable to the system, until the last one when the payload is generated. In the figure we can also notice that each module processes the data and passes the result not only to the next module, but also to a specific namespace in the SDB, corresponding to each module, for storage.

Another important aspect is the SMT Solver Module, in the diagram we can see that this is composed of two logical parts, the actual SMT Solver which is a standalone entity, and a wrapper. The SMT Wrapper is the one responsible with the generation of possible payloads. It in turn uses the SMT Solver to test these payloads until it finds a satisfiable one.

\fig[scale=0.4]{src/img/pipeline.png}{img:pipeline}{Work Flow Diagram}

\section{Radare2 Core}

The Radare2 Core represents a key component of the project as it provides access to all the tools and features already implemented in the framework. One of the main features provided by the Core is the ROP gadget search tool as well as access to the SDB and interfacing all modules.

Note that each module or tool can be called individually and in any combination provided the prerequisites for them are respected.

\section{Gadget Classification Module}

This module provides the semantic analysis and classification for each gadget. The raw gadgets are generated by the ROP search feature already implemented in the Core and fed to the module. Each gadget is classified into one or more categories as described in \labelindexref{Table}{table:gadgetTypes}.

\section{SMTLib2 Translation Module}

The SMTLib2 Translation Module implements a parser for translating ESIL to the SMTLib2 format\cite{barret2015smtlib}. This module takes the output from the previous classification module and translates it. By introducing this translation module for SMTLib2 format between the classification step and actual SMT Solver, the user has the possibility to plug in any SMT Solver which uses this standard, instead of having to individually translate the input for custom solvers.

\section{Payload Generation Module}

If the SMT Solver finds the formula satisfiable it generates a payload for the specific executable in question. The module determines which gadgets were used to satisfy the target formula and generates a payload with their corresponding addresses.

\section{SDB}

Each of the previously described modules use the String Data Base present in Radare2 to store their results. The modules first check the SDB to see if the same task was already processed and perform a quick lookup before processing it. The user can query the results of each individual step and inspect or modify it, as well as save and restore the results into projects and use them across different sessions.

